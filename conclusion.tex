\chapter{Conclusion}
\label{chap:conclusion}

% refer back to introduction - what was the aim
% the LHC delivering first significant 13TeV dataset
The collection of a significant dataset of $\sqrt{s}=13~\tev$
$pp$-collision data by the \LHC during the early stages of Run~2 has
presented the best opportunity in recent years to explore \BSM physics
at a new energy frontier. With all the particles of the \SM
discovered, but a series of serious questions unanswered, there is a
strong motivation to search for \BSM physics. One of the most
promising \BSM theories, natural \SUSY, is now well within reach of
the particle physics community. A search for this in the
sensitive all-hadronic final state has been carried out and presented
within this thesis.

% to maintain performance dadadada...
% trigger work and improvements
The start of Run~2 saw the \LHC move into a new and challenging
regime. With a higher collision energy and the promise of a
significant increase in \PU, the rate of production of high energy
physics processes increased substantially. This presented a
big challenge for the Level-1 hardware trigger. The hardware itself has been upgraded and a new Level-1 jet
algorithm with dynamic \PUS has been devised and implemented. This
algorithm has had significant success in meeting the challenges of
Run~2. It has been commissioned is currently
in operation, allowing low energy thresholds on the types of hadronic
events that are collected. This is particularly important for the analyses at \CMS that
target \SUSY.

% alphaT analysis
% summary of the signal region selections used
% summary of background prediction
% summary of rigorous systematics robust to early data
% the result
A search for \SUSY in the all-hadronic final state has been described
and the results of the search with $12.9\pm0.8~\ifb$ of Run~2
collision data has been presented. The analysis revolves around
suppressing the dominant \QCD multijet background with requirements
made on the \alphat and \bdphi topological variables. A data-driven
prediction of the multijet background demonstrates that it
is reduced to a negligible level. The remaining backgrounds 
mainly constitute \SM processes that produce neutrinos, a
source of genuine \MET. These backgrounds are predicted through a
data-driven method that makes use of muon and photon control samples.
The predicted background yields in the signal region are obtained by
extrapolating from the control regions through transfer factors made
from
simulation. Appropriate systematic uncertainties on the magnitude of
these transfer factors and the shape of the \MHT dimension, which is
taken directly from simulation, are calculated. This allows the final result to
be obtained through a maximum-likelihood fit for a series of signal
hypotheses.

The search has found that the observed data within the signal region
is compatible with the expected yields from the \SM backgrounds. As no
evidence for \BSM phenomena is observed, limits are set on the
production of a series of key \SUSY models. These results extend
beyond the limits set during Run~1 of the \LHC. The gluino mass is
excluded up to $1775~\gev$, the sbottom mass is excluded up to
$1025~\gev$ and the stop mass is excluded up to $875~\gev$ with model
dependent exclusions of the \LSP in the range of a few hundred \gev.

% no evidence so far
% look to the future - bigger data sets, DM interpretation
% other options for DM! how is SUSY hiding?
% future of the LHC
With the results of hadronic searches for \BSM physics not observing
any evidence of non-\SM phenomena, the case for natural \SUSY is
starting to weaken. If it were to exist in one of its most commonly
conceived forms, it would be expected that the high energy run of the
\LHC would start to observe evidence of the existence of such \SUSY
particles. It is still possible that with a larger dataset and a
better understanding of the systematic effects in the data, \SUSY could present
itself at the \LHC. However, the best opportunity so far for a glimpse of
natural \SUSY has resulted in a null result. 

There is clear evidence for the existence of \DM in the universe, with
a weakly interacting massive particle remaining a favoured candidate.
The hadronic \SUSY searches are in a prime position to reinterpret
their results from the perspective of these sort of generic \DM
models. The \LHC will therefore remain one of the most powerful tools
in probing the energy frontier beyond the \SM.

Now is a pivotal moment in the history of particle physics. As the
\LHC continues to collect high energy collision data and the
understanding of the data collected by \CMS matures, there will be
more concrete conclusions about the nature of \BSM physics.  It will
be possible to finally discount the existence of favoured forms of
natural \SUSY in the next couple of years of \LHC data. There is now
the potential to observe any manner of other, more difficult to observe,
\BSM scenarios that may exist at the new energy frontier. If nothing
presents itself by the end of Run~2, the field will then have to
reassess exactly how best to search for \BSM phenomena and how to move
into the future.

