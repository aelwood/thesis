\chapter{Theory}
\label{chap:theory}

% \chapterquote{NO FATE BUT THE NARRATIVES WE IMPOSE ON LIFE'S RANDOM CHAOS TO
% DISTRACT OURSELVES FROM OUR EXISTENTIAL PLIGHT}{xkcd 1177}

\section{The Standard Model of particle physics}
\label{sec:sm}

%andrew's thesis looks quite brief and concise
The \acf{SM} describes the interaction of matter through the
electromagnetic, weak nuclear and strong forces in the context of a
renormalisable quantum field theory
\cite{Salam:1964ry,Glashow:1961tr,PhysRevLett.19.1264}. The matter
particles are represented as spin-$\frac{1}{2}$ fermionic fields and
forces are represented as spin-1 bosonic fields. An additional spin-0
\emph{Higgs} field is included to provide the bosons of the weak force
with their mass. A key principle in the \SM is that of local gauge
invariance. The gauge group of the \SM is $SU(3)\times SU(2) \times
U(1)$, when these symmetries are applied to the matter particles they
imply the presence of the force carrying bosons. How this leads to the
particle phenomenology of the \SM will be briefly explored in this
section.

Throughout this section the convention $c=\hbar = 1$ is used and the
Einstein vector summation convention is assumed. Four vector indices
are labelled as $\mu$ and $\nu$.

\subsection{The fundamental particles}

The fundamental particles of the \SM comprise
fermions and the force mediating bosons, a summary them and their
relevant charges can be seen in Table~\ref{tab:smParticles}. 

The fermions consist of three generations of charged leptons and their
corresponding weak force partner, the neutrinos. There are
additionally three generations of up-quarks and down-quarks. For all
of these twelve fermions there are corresponding antiparticles that
have the same mass but opposite quantum numbers.

As fermions are spin-$\frac{1}{2}$ particles they are described by the
Dirac equation \cite{Griffiths:111880}:
\begin{equation}

\end{equation}

There are five types of bosons that arise from the \SM gauge
symmetries: photons, gluons, $W^{\pm}$, $Z^0$ and the Higgs. Their
properties will be discussed further in this section.

\subsection{Gauge symmetries}

\begin{table}[htbp!]
\begin{tabular}{l|l c|c|c|c|c|c}
%\hline 
Categories & Particle & & Mass & Spin & Electric & Colour & Weak \\
 &  & & & & charge & charge & isospin \\
\hline
\hline
Leptons & electron & $e$ & 0.511~MeV & & & & \\
                & muon & $\mu$ & 106~MeV & $\frac{1}{2}$ & -1  & 0 &  $-\frac{1}{2}$\\
                & tau & $\tau$ & 1777~MeV  & &   &  &  \\
\hline
Neutrinos & electron & $\nu_e$ & <225~eV & & & & \\
                & muon & $\nu_{\mu}$ & <0.19~MeV & $\frac{1}{2}$  & 0  & 0 &  $+\frac{1}{2}$\\
                & tau & $\nu_{\tau}$ & <18.2~MeV &  &   &  &  \\
\hline
Up quarks & up & $u$ & 2.3~MeV & & & \\
                & charm & $c$ & 1.28~GeV & $\frac{1}{2}$ & $+\frac{2}{3}$  & $r,g,b$ &  $+\frac{1}{2}$\\
                & top & $t$ & 173~GeV  &   &  &  \\
\hline
Down quarks & down & $d$ & 4.8~MeV & & & & \\
                & strange & $s$ & 95~MeV & $\frac{1}{2}$ & $-\frac{1}{3}$  & $r,g,b$ &  $-\frac{1}{2}$\\
                & bottom & $b$ & 4.18~GeV &  &   &  &  \\
\hline
Force & photon & $\gamma$ & 0 &  & 0 & 0 & 0 \\
\cline{2-8}
 mediating &  &  &  &  &  & $r\bar{g},r\bar{b},g\bar{r},g\bar{b}$\\
 bosons & gluon & $g$ & 0 &  & 0 & $b\bar{r},b\bar{g},
\frac{1}{\sqrt{2}}(r\bar{r}-g\bar{g})$ & 0 \\
 & &  &  &  &  & $\frac{1}{\sqrt{6}}(r\bar{r}+g\bar{g}-2b\bar{b})$ &  \\
\cline{2-8}
 & W & $W^{\pm}$ & 80.4~GeV & 1 & $\pm 1$ & 0 & $\pm 1$ \\
 & Z & $Z^0$ & 91.2~GeV &  & 0 & 0 & 0 \\
 & Higgs & $h^0$ & 125~GeV &  & 0 & 0 & $-\frac{1}{2}$ \\
%\hline
\end{tabular}
\caption{All the fundamental Standard Model fermions and bosons and their charges
\cite{PhysRevD.86.010001}}
\label{tab:smParticles}
\end{table}

The 

\subsection{The strong force}

%describe QCD
%include some jet phenomenology

\subsection{Electroweak unification}

%describe EWK

%%%%%%%% NOTES %%%%%%%

%fundamental particles, leptons and photons and the charges they carry
% show a table of the particles

%technical again SU2 ..... where the components are
% which forces these lead to

%ewk unification etc.

%put a bit about jet phenomenology from a QCD point of view

%%%%%%%%% LEAVING THESE OUT FOR NOW %%%%%%%%%%%%%
%\subsection{Gauge symmetries}
%specific example of gauge symmetries put here by Mark and Robyn
%\subsection{Electroweak unification}
%only need this section if i put gauge symmetries
%prefer to leave these out and be more technical and concise

\subsection{The Higgs mechanism}

\subsection{Beyond the Standard Model}

% all of SM verified so far, everything found, (requires some
% modifications for neutrino masses)
% but there are bigger problems:

%hierarchy problem

%DM
% show the bullet cluster?

%GUT

\section{Supersymmetry}
\label{sec:susy}

%fundamental spacetime symmetry

% as in Mark's and in primer

%solves above problems

\subsection{The Minimal Supersymmetric Standard Model (MSSM)}

\subsection{Signatures of supersymmetry at the \LHC}

As the \LHC is a hadron collider, the highest cross section SUSY
production processes occur via the strong force
\cite{Martin:1997ns}
\cite{SUSYxsections_NewAspectsof_pp_collisions}. These processes
result in the production of squarks and gluinos, the SUSY particles
with colour charge. In all favoured SUSY models, these relatively
heavy particles decay within the detector to a weakly interacting \LSP,
usually a neutralino \cite{Farrar:1978xj}. In
collisions at the \LHC this will appear as several hard jets with
unbalanced momentum (missing energy). 

\subsection{Simplified models}

%Decays etc.
