\chapter{Theory}
\label{chap:theory}

% \chapterquote{NO FATE BUT THE NARRATIVES WE IMPOSE ON LIFE'S RANDOM CHAOS TO
% DISTRACT OURSELVES FROM OUR EXISTENTIAL PLIGHT}{xkcd 1177}

\section{The Standard Model of particle physics}
\label{sec:sm}

%andrew's thesis looks quite brief and concise
The \acf{SM} describes the interaction of matter through the
electromagnetic, weak nuclear and strong forces in the context of a
renormalisable quantum field theory
\cite{Salam:1964ry,Glashow:1961tr,PhysRevLett.19.1264}. The matter
particles are represented as spin-$\frac{1}{2}$ fermionic fields and
forces are represented as spin-1 bosonic fields. An additional spin-0
\emph{Higgs} field is included to provide the bosons of the weak force
with their mass. A key principle in the \SM is that of local gauge
invariance. The gauge group of the \SM is $SU(3)\times SU(2) \times
U(1)$, when these symmetries are applied to the matter particles they
imply the presence of the force carrying bosons. How this leads to the
particle phenomenology of the \SM will be briefly explored in this
section.

The \SM is typically considered within a lagrangian formalsim. In a
quantum field theory all the the relevant fields and their
interactions are described by a lagrangian density. The lagrangian
density of the \SM can be divided into four parts, which will each be
addressed within this section:
\begin{equation}
\mathcal{L}_{SM}=\mathcal{L}_{gauge}+\mathcal{L}_{fermion}+\mathcal{L}_{Higgs}+\mathcal{L}_{Yukawa}.
\end{equation}

Throughout this section the convention $c=\hbar = 1$ is used and the
Einstein four-vector summation convention is assumed. Four-vector indices
are labelled as $\mu$ and $\nu$.

\subsection{The fundamental particles}

The fundamental particles of the \SM comprise
fermions and the force mediating bosons, a summary them and their
relevant charges can be seen in Table~\ref{tab:smParticles}. 

The fermions consist of three generations of charged leptons and their
corresponding weak force partner, the neutrinos. There are
additionally three generations of up-quarks and down-quarks. For all
of these twelve fermions there are corresponding antiparticles that
have the same mass but opposite quantum numbers. As fermions are
spin-$\frac{1}{2}$ particles they are described by the Dirac equation
\cite{Griffiths:111880}:
\begin{equation}
(i\gamma^{\mu}\partial_{\mu}-m)\psi=0,
\end{equation}
where $\gamma^{\mu}$ are the Dirac matrices, defined by their
anti-commutation relation:
\begin{equation}
\{\gamma^{\mu},\gamma^{\nu}\}=\gamma^{\mu}\gamma^{\nu}+\gamma^{\nu}\gamma^{\mu}=2g^{\mu\nu},
\end{equation}
where $g^{\mu\nu}$ is the Minkowski metric. The covariant derivative is denoted
by $\partial_{\mu}$ and $m$ is the mass of the particle in question.

There are five types of bosons that arise from the \SM gauge
symmetries: photons, gluons, $W^{\pm}$, $Z^0$ and the Higgs. Their
properties will be discussed further in this section.

\begin{table}[htbp!]
\begin{tabular}{l|l c|c|c|c|c|c}
%\hline 
Categories & Particle & & Mass & Spin & Electric & Colour & Weak \\
 &  & & & & charge & charge & isospin \\
\hline
\hline
Leptons & electron & $e$ & 0.511~MeV & & & & \\
                & muon & $\mu$ & 106~MeV & $\frac{1}{2}$ & -1  & 0 &  $-\frac{1}{2}$\\
                & tau & $\tau$ & 1777~MeV  & &   &  &  \\
\hline
Neutrinos & electron & $\nu_e$ & <225~eV & & & & \\
                & muon & $\nu_{\mu}$ & <0.19~MeV & $\frac{1}{2}$  & 0  & 0 &  $+\frac{1}{2}$\\
                & tau & $\nu_{\tau}$ & <18.2~MeV &  &   &  &  \\
\hline
Up quarks & up & $u$ & 2.3~MeV & & & \\
                & charm & $c$ & 1.28~GeV & $\frac{1}{2}$ & $+\frac{2}{3}$  & $r,g,b$ &  $+\frac{1}{2}$\\
                & top & $t$ & 173~GeV  &   &  &  \\
\hline
Down quarks & down & $d$ & 4.8~MeV & & & & \\
                & strange & $s$ & 95~MeV & $\frac{1}{2}$ & $-\frac{1}{3}$  & $r,g,b$ &  $-\frac{1}{2}$\\
                & bottom & $b$ & 4.18~GeV &  &   &  &  \\
\hline
Force & photon & $\gamma$ & 0 &  & 0 & 0 & 0 \\
\cline{2-8}
 mediating &  &  &  &  &  & $r\bar{g},r\bar{b},g\bar{r},g\bar{b}$\\
 bosons & gluon & $g$ & 0 &  & 0 & $b\bar{r},b\bar{g},
\frac{1}{\sqrt{2}}(r\bar{r}-g\bar{g})$ & 0 \\
 & &  &  &  &  & $\frac{1}{\sqrt{6}}(r\bar{r}+g\bar{g}-2b\bar{b})$ &  \\
\cline{2-8}
 & W & $W^{\pm}$ & 80.4~GeV & 1 & $\pm 1$ & 0 & $\pm 1$ \\
 & Z & $Z^0$ & 91.2~GeV &  & 0 & 0 & 0 \\
 & Higgs & $h^0$ & 125~GeV &  & 0 & 0 & $-\frac{1}{2}$ \\
%\hline
\end{tabular}
\caption{All the fundamental Standard Model fermions and bosons and
their charges \cite{PhysRevD.86.010001}}
\label{tab:smParticles}
\end{table}

\subsection{Gauge symmetries}

The insensitivity to the structure of a theory to a specific
transformation constitutes a symmetry. This concept is very powerful
for gaining insights into fundamental physical theories. For example,
the fact that physical laws do not change over time, time
translational symmetry, leads to the conservation of energy.  In
general, any symmetries have a corresponding conserved quantity, as
laid out in Noether's theorem \cite{1971TTSP....1..186N}. This concept
is used extensively when formulating the \SM and allows for the
derivation of observed interactions through the imposition of a few,
fairly straightforward, symmetries.

The effect of applying a symmetry within the \SM is demonstrated when
imposing local $U(1)$ invariance on the Dirac lagrangian for a
fermion, with wave function $\psi$ and mass, $m$ \cite{Griffiths:111880}:
\begin{equation}
\mathcal{L}=i\bar{\psi}\cancel{\partial}\psi-m\bar{\psi}\psi.
\end{equation}
A global $U(1)$ transformation, $\psi\rightarrow e^{iq\theta}\psi$,
where the phase $\theta$ and $q$ are constant,
leaves the lagrangian invariant. If this $U(1)$ transformation is
local, i.e. the phase depends on spacetime position, $x$, then the
lagrangian is no longer invariant. It now transforms as:
\begin{equation}
\label{eq:uninvariance}
\mathcal{L}\rightarrow\mathcal{L}-q(\partial_{\mu}\theta(x))\bar{\psi}\gamma^{\mu}\psi.
\end{equation}
However, one can add a vector field, $A_{\mu}$, that interacts with
the fermion field through the lagrangian term:
\begin{equation}
\mathcal{L}_{int}=q(\bar{\psi}\gamma^{\mu}\psi) A_{\mu},
\end{equation}
This vector field is chosen to transform as $A_{\mu}\rightarrow
A_{\mu}+\partial_{\mu}\theta$ and is known as a \emph{gauge field} or
\emph{gauge boson}. The interaction lagrangian term then transforms
under a local gauge transformation as:
\begin{equation}
\mathcal{L}_{int}\rightarrow \mathcal{L}_{int}+q(\partial_{\mu}\theta)\bar{\psi}\gamma^{\mu}\psi,
\end{equation}
this cancels out the term that violated local gauge invariance in
Equation~\ref{eq:uninvariance}. The existence of a new gauge field
allows the addition of an additional gauge invariant term containing
the field strength tensor of the vector field, $F_{\mu\nu}$, which can
be written in general as:
\begin{equation}
F_{\mu\nu}^a=\partial_{\mu}A_{\nu}^a-\partial_{\nu}A_{\mu}^a+gf_{abc}A_{\mu}^{b}A_{\nu}^{c},
\end{equation}
for a general gauge group with the structure constants $f^{abc}$. For
the $U(1)$ group there is only one self-commuting generator so the
structure constant is 0. For non-Abelian gauge groups, such as
$SU(3)$, the structure constants can be non-zero, which introduces
self interaction terms within the lagrangian. In this case the gauge
boson will carry a charge and can interact with itself.

The final lagrangian for a Dirac fermion can then be written as:
\begin{equation}
  \label{eq:localdiraclagrangian}
  \mathcal{L}=i\bar{\psi}\gamma^{\mu}\mathcal{D}_{\mu}\psi-m\bar{\psi}\psi-\frac{1}{4}F_{\mu\nu}F^{\mu\nu},
\end{equation}
where $\mathcal{D}_{\mu}=\partial_{\mu}+iqA_{\mu}$ and is known as the
\emph{covariant derivative}. This lagrangian will be invariant under
local $U(1)$ transformations. In this case, the addition of one extra
gauge field maintains local invariance as $U(1)$ transformations have
one degree of freedom, corresponding to a generator of the group.
The local gauge invariance of symmetries with more degrees of freedom
require the addition of one gauge boson per degree of freedom. 

The general principle of obtaining local gauge invariance method
through gauge bosons is applied with great success to the gauge group
of the \SM. With the choice of an appropriate gauge group,
$SU(3)\times SU(2) \times U(1)$, the bosons that describe the strong,
weak and electromagnetic forces can all be obtained. In the example
demonstrated in this section the final lagrangian
(Eq.~\ref{eq:localdiraclagrangian}) is that which describes \ac{QED},
predicting the massless photon field, $A_{\mu}$, from the $U(1)$ local
gauge invariance of fermions with a coupling strength corresponding to
the electric charge, represented by $q$ in the above. % check this!

\subsection{The strong force}

The strong force can be described by the $SU(3)$ gauge group,
resulting in a description of the interaction between quark fields
via eight massless gluon gauge fields. This theory is known as \QCD and
the quark fields possess a colour charge, $C=(r,g,b)$. As the
$SU(3)$ group is non-Abelian, the gluons also possess a colour and
anti-colour charge. This leads to gluon-self couplings, which
results in the short range of the strong force. Additionally,
screening effects from virtual gluons, carrying the colour charge,
leads to the phenomena known as \emph{asymptotic freedom}
\cite{PhysRevLett.30.1343}. The strong coupling constant, $\alpha_S$
gets weaker over short ranges, which leads to quarks behaving as if
they are unbound when they are very close but more strongly coupled as
they move part. The fact that $\alpha_S$ can be small makes it very
challenging to calculate \QCD perturbatively, using the well known
techniques that work for electromagnetic and weak force calculations.

One significant consequence of the strong force is that quarks are
confined to exist in colour-singlet states. This can lead to either
\emph{mesons} that consist of a quark-antiquark pair, which have the
same colour and anti-colour charge, or \emph{baryons} that are a
triple quark or anti-quark bound state with each quark having one of
the three different colour charges. These quark bound states are known
collectively as \emph{hadrons}. Despite being very strongly bound, if
quarks in these bound states are given significant energy they can be
liberated through the production of quark-antiquark pairs from the
vacuum. This process is known as \emph{hadronisation} and occurs when
the energy contained within the gluons binding the quarks more than
the energy contained within the mass of the produced hadron. When this
occurs in the environment of a particle collider the quarks typically
gain a significant momentum. This causes them to move along the
momentum direction out of the bound state, undergoing hadronisation
and fragmentation of the hadrons that still have significant energy.
This leads to a quark producing a collimated emission of hadrons from
the collision point, known as a \emph{jet}.

\subsection{Electroweak unification}

%describe EWK

%%%%%%%% NOTES %%%%%%%

%fundamental particles, leptons and photons and the charges they carry
% show a table of the particles

%technical again SU2 ..... where the components are
% which forces these lead to

%ewk unification etc.

%put a bit about jet phenomenology from a QCD point of view

%%%%%%%%% LEAVING THESE OUT FOR NOW %%%%%%%%%%%%%
%\subsection{Gauge symmetries}
%specific example of gauge symmetries put here by Mark and Robyn
%\subsection{Electroweak unification}
%only need this section if i put gauge symmetries
%prefer to leave these out and be more technical and concise

\subsection{The Higgs mechanism}

\subsection{Beyond the Standard Model}

% all of SM verified so far, everything found, (requires some
% modifications for neutrino masses)
% but there are bigger problems:

%hierarchy problem

%DM
% show the bullet cluster?

%GUT

\section{Supersymmetry}
\label{sec:susy}

%fundamental spacetime symmetry

% as in Mark's and in primer

%solves above problems

\subsection{The Minimal Supersymmetric Standard Model (MSSM)}

\subsection{Signatures of supersymmetry at the \LHC}

As the \LHC is a hadron collider, the highest cross section SUSY
production processes occur via the strong force
\cite{Martin:1997ns}
\cite{SUSYxsections_NewAspectsof_pp_collisions}. These processes
result in the production of squarks and gluinos, the SUSY particles
with colour charge. In all favoured SUSY models, these relatively
heavy particles decay within the detector to a weakly interacting \LSP,
usually a neutralino \cite{Farrar:1978xj}. In
collisions at the \LHC this will appear as several hard jets with
unbalanced momentum (missing energy). 

\subsection{Simplified models}

%Decays etc.
