\chapter{Theory}
\label{chap:theory}

% \chapterquote{NO FATE BUT THE NARRATIVES WE IMPOSE ON LIFE'S RANDOM CHAOS TO
% DISTRACT OURSELVES FROM OUR EXISTENTIAL PLIGHT}{xkcd 1177}

\section{The Standard Model of particle physics}
\label{sec:sm}

%andrew's thesis looks quite brief and concise
The \acf{SM} describes the interaction of matter through the
electromagnetic, weak nuclear and strong forces in the context of a
renormalisable quantum field theory
\cite{Salam:1964ry,Glashow:1961tr,PhysRevLett.19.1264}. The matter
particles are represented as spin-$\frac{1}{2}$ fermionic fields and
forces are represented as spin-1 bosonic fields. An additional spin-0
\emph{Higgs} field is included to provide the bosons of the weak force
with their mass. The \SM is built around the concept of local gauge
invariance. When the symmetries of the \SM local gauge group,
$SU(3)\times SU(2) \times U(1)$, are applied to the fermions they
imply the existence of the force carrying bosons. This section will
briefly explore how this leads to the particle phenomenology of the
\SM.

The \SM is typically considered within a lagrangian formalsim. In a
quantum field theory all the the relevant fields and their
interactions are described by a lagrangian density. The lagrangian
density of the \SM can be divided into four parts:
\begin{equation}
\mathcal{L}_{SM}=\mathcal{L}_{gauge}+\mathcal{L}_{fermion}+\mathcal{L}_{Higgs}+\mathcal{L}_{Yukawa},
\end{equation}
$\mathcal{L}_{fermion}$ describes the fermion
fields, their interactions with the bosons are described in
$\mathcal{L}_{gauge}$. The final two terms, $\mathcal{L}_{Higgs}$ and
$\mathcal{L}_{Yukawa}$, describe how the particles within the \SM
obtain mass through interactions with the Higgs field.

Throughout this section the convention $c=\hbar = 1$ is used and the
Einstein four-vector summation convention is assumed. Four-vector indices
are labelled as $\mu$ and $\nu$.

\subsection{The fundamental particles}

The fundamental particles of the \SM comprise fermions and the force
mediating bosons, a summary of them and their relevant
electromagnetic, weak and strong force charges can be seen in
Table~\ref{tab:smParticles}. 

The fermions consist of three generations of charged leptons and their
corresponding weak force partner, the neutrinos. There are
additionally three generations of up-quarks and down-quarks. For all
of these twelve fermions there are corresponding antiparticles that
have the same mass but opposite quantum numbers. As fermions are
spin-$\frac{1}{2}$ particles they are described by the Dirac equation
\cite{Griffiths:111880}:
\begin{equation}
(i\gamma^{\mu}\partial_{\mu}-m)\psi=0,
\end{equation}
where $\gamma^{\mu}$ are the Dirac matrices, defined by their
anti-commutation relation:
\begin{equation}
\{\gamma^{\mu},\gamma^{\nu}\}=\gamma^{\mu}\gamma^{\nu}+\gamma^{\nu}\gamma^{\mu}=2g^{\mu\nu},
\end{equation}
where $g^{\mu\nu}$ is the Minkowski metric. The covariant derivative is denoted
by $\partial_{\mu}$ and $m$ is the mass of the particle in question.

There are five types of bosons that arise from the \SM gauge
symmetries: photons, gluons, $W^{\pm}$, $Z^0$ and the Higgs. Their
properties will be discussed further in this section.

%\begin{table}[htbp!]
\begin{table}
\begin{tabular}{l|l c|c|c|c|c|c}
%\hline 
Categories & Particle & & Mass & Spin & Electric & Colour & Weak \\
 &  & & & & charge & charge & isospin ($t_3$)\\
\hline
\hline
Leptons & electron & $e$ & 0.511~MeV & & & & \\
                & muon & $\mu$ & 106~MeV & $\frac{1}{2}$ & -1  & 0 &  $-\frac{1}{2}$\\
                & tau & $\tau$ & 1777~MeV  & &   &  &  \\
\hline
Neutrinos & electron & $\nu_e$ & <225~eV & & & & \\
                & muon & $\nu_{\mu}$ & <0.19~MeV & $\frac{1}{2}$  & 0  & 0 &  $+\frac{1}{2}$\\
                & tau & $\nu_{\tau}$ & <18.2~MeV &  &   &  &  \\
\hline
Up- & up & $u$ & 2.3~MeV & & & \\
type      & charm & $c$ & 1.28~GeV & $\frac{1}{2}$ & $+\frac{2}{3}$  & $r,g,b$ &  $+\frac{1}{2}$\\
quarks          & top & $t$ & 173~GeV  &   &  &  \\
\hline
Down-  & down & $d$ & 4.8~MeV & & & & \\
type & strange & $s$ & 95~MeV & $\frac{1}{2}$ & $-\frac{1}{3}$  & $r,g,b$ &  $-\frac{1}{2}$\\
quarks            & bottom & $b$ & 4.18~GeV &  &   &  &  \\
\hline
Force & photon & $\gamma$ & 0 & 1 & 0 & 0 & 0 \\
\cline{2-8}
 mediating &  &  &  &  &  & $r\bar{g},r\bar{b},g\bar{r},g\bar{b}$\\
 bosons & gluon & $g$ & 0 & 1 & 0 & $b\bar{r},b\bar{g},
\frac{1}{\sqrt{2}}(r\bar{r}-g\bar{g})$ & 0 \\
 & &  &  &  &  & $\frac{1}{\sqrt{6}}(r\bar{r}+g\bar{g}-2b\bar{b})$ &  \\
\cline{2-8}
 & W & $W^{\pm}$ & 80.4~GeV &  & $\pm 1$ &  & $\pm 1$ \\
 & Z & $Z^0$ & 91.2~GeV & 1 & 0 & 0 & 0 \\
 & Higgs & $h^0$ & 125~GeV &  & 0 &  & $-\frac{1}{2}$ \\
%\hline
\end{tabular}
\caption{All the fundamental Standard Model fermions and bosons and
their charges \cite{PhysRevD.86.010001}}
\label{tab:smParticles}
\end{table}

\subsection{Gauge symmetries}
\label{sec:gaugeSymmetries}

The insensitivity to the structure of a theory to a specific
transformation constitutes a symmetry. This concept is very powerful
for gaining insights into fundamental physical theories. For example,
the fact that physical laws do not change over time,
time-translational symmetry, leads to the conservation of energy.  In
general, any symmetries have a corresponding conserved quantity, as
laid out in Noether's theorem \cite{1971TTSP....1..186N}. This concept
is used extensively when formulating the \SM and allows for the
derivation of observed interactions through the imposition of a few,
fairly straightforward, symmetries.

The effect of applying a symmetry within the \SM is demonstrated when
imposing local $U(1)$ invariance on the Dirac lagrangian for a
fermion, with wavefunction $\psi$ and mass $m$ \cite{Griffiths:111880}:
\begin{equation}
\mathcal{L}=i\bar{\psi}\cancel{\partial}\psi-m\bar{\psi}\psi.
\end{equation}
A global $U(1)$ transformation, $\psi\rightarrow e^{iq\theta}\psi$,
where the phase $\theta$ and $q$ are constant,
leaves the lagrangian invariant. If this $U(1)$ transformation is
local, i.e. the phase depends on spacetime position, $x$, then the
lagrangian is no longer invariant. It now transforms as:
\begin{equation}
\label{eq:uninvariance}
\mathcal{L}\rightarrow\mathcal{L}-q(\partial_{\mu}\theta(x))\bar{\psi}\gamma^{\mu}\psi.
\end{equation}
However, one can add a vector field, $A_{\mu}$, that interacts with
the fermion field through the lagrangian term:
\begin{equation}
\mathcal{L}_{int}=q(\bar{\psi}\gamma^{\mu}\psi) A_{\mu},
\end{equation}
This vector field is chosen to transform as $A_{\mu}\rightarrow
A_{\mu}+\partial_{\mu}\theta$ and is known as a \emph{gauge field} or
\emph{gauge boson}. The interaction lagrangian term then transforms
under a local gauge transformation as:
\begin{equation}
\mathcal{L}_{int}\rightarrow \mathcal{L}_{int}+q(\partial_{\mu}\theta)\bar{\psi}\gamma^{\mu}\psi,
\end{equation}
this cancels out the term that violated local gauge invariance in
Equation~\ref{eq:uninvariance}. The existence of a new gauge field
allows the addition of an additional gauge invariant term containing
the field strength tensor of the vector field, $F_{\mu\nu}$, which can
be written in general as:
\begin{equation}
F_{\mu\nu}^a=\partial_{\mu}A_{\nu}^a-\partial_{\nu}A_{\mu}^a+gf_{abc}A_{\mu}^{b}A_{\nu}^{c},
\end{equation}
for a general gauge group with the structure constants $f^{abc}$ and
self-coupling constant $g$. For
the $U(1)$ group there is only one self-commuting generator so the
structure constant is 0. For non-Abelian gauge groups, such as
$SU(3)$, the structure constants can be non-zero, which introduces
self interaction terms within the lagrangian. In this case the gauge
boson will carry a charge and can interact with itself.

The final lagrangian for a Dirac fermion can then be written as:
\begin{equation}
  \label{eq:localdiraclagrangian}
  \mathcal{L}=i\bar{\psi}\gamma^{\mu}\mathcal{D}_{\mu}\psi-m\bar{\psi}\psi-\frac{1}{4}F_{\mu\nu}F^{\mu\nu},
\end{equation}
where $\mathcal{D}_{\mu}=\partial_{\mu}+iqA_{\mu}$ and is known as the
\emph{covariant derivative}. This lagrangian will be invariant under
local $U(1)$ transformations. In this case, the addition of one extra
gauge field maintains local invariance as $U(1)$ transformations have
one degree of freedom, corresponding to a generator of the group.
The local gauge invariance of symmetries with more degrees of freedom
require the addition of one gauge boson per degree of freedom. 

The general principle of obtaining local gauge invariance method
through gauge bosons is applied with great success to the gauge group
of the \SM. With the choice of an appropriate gauge group,
$SU(3)\times SU(2) \times U(1)$, the bosons that describe the strong,
weak and electromagnetic forces can all be obtained. In the example
demonstrated in this section the final lagrangian
(Eq.~\ref{eq:localdiraclagrangian}) is that which describes \ac{QED},
predicting the massless photon field, $A_{\mu}$, from the $U(1)$ local
gauge invariance of fermions with a coupling strength corresponding to
the electric charge, represented by $q$. % check this!

\subsection{The strong force}

The strong force can be described with the $SU(3)$ gauge group,
resulting in the interaction of quark fields
via eight massless gauge fields, the gluons. This theory is known as \QCD and
the quark fields possess a colour charge, $C=(r,g,b)$. As the
$SU(3)$ group is non-Abelian, the gluons also possess a colour and
anti-colour charge. This leads to gluon-self couplings, which
results in the short range of the strong force. Additionally,
screening effects from virtual gluons, carrying the colour charge,
leads to the phenomena known as \emph{asymptotic freedom}
\cite{PhysRevLett.30.1343}. This is characterised by the strong coupling constant, $\alpha_S$
getting weaker over short ranges, which leads to quarks behaving as if
they are unbound when they are very close but more strongly coupled as
they move part. The fact that $\alpha_s$ can be small makes it very
challenging to calculate \QCD perturbatively using the well known
techniques that work for electromagnetic and weak force calculations.

One important property of the strong force is that quarks are
confined to exist in colour-singlet states. This can lead to either
\emph{mesons} comprising a quark-antiquark pair or \emph{baryons} that are a
triple quark or anti-quark bound state. These different bound states are known
collectively as \emph{hadrons}. Despite being very strongly bound, if
quarks in these states are given significant energy they can be
liberated through the pair production of quark-antiquark pairs. This
process is known as \emph{hadronisation} and occurs when the energy
contained within the gluons binding the quarks more than the energy
contained within the mass of the produced hadron. When this occurs in
the environment of a particle collider, the quarks typically gain a
significant momentum. This causes them to move out of the bound state,
undergoing hadronisation. If the hadrons that are produced also have
significant energy they can also undergo further fragmentation. This
leads to a quark producing a collimated emission of hadrons from
the collision point, known as a \emph{jet}.

\subsection{Electroweak unification}

The electromagnetic and weak forces are described in the \SM by the
symmetry group $SU(2)\times U(1)$. The requirement of local gauge
invariance in the weak sector led to the electromagnetic and weak
forces being unified within this group in a landmark achievement in
the 1960s \cite{Glashow:1961tr,PhysRevLett.19.1264,Salam:1964ry}.

The $SU(2)$ group has three generators, $T_i=\tau_i/2$, where
$i=1,2,3$ and $\tau_i$ are the Pauli spin matrices. As mentioned in
Sec.~\ref{sec:gaugeSymmetries}, each of these generators is manifested
as a gauge field, labelled $W_{\mu}^i$. Within the electroweak theory
these gauge fields only act on the left handed chiral component of the
fermion field, $\psi_L$, where $\psi_L = (1-\gamma_5)\psi$ and
$\gamma^5=i\gamma^0\gamma^1\gamma^2\gamma^3$. This 
\emph{left-handedness} of the electroweak theory leads to the parity violation
that is observed in weak interactions. The charges associated with
these gauge fields are known as \emph{weak isospin} and are denoted
$t_i$. As with the $SU(3)$ group the $SU(2)$ group is non-Abelian and
the gauge bosons are able to interact with themselves.

The $U(1)$ group has a single generator, with an associated gauge
field, $B_{\mu}$. This field interacts with particles that
carry \emph{weak hypercharge}, $Y=2(Q-t_3)$. It is worth noting that this
is a different charge to that of the $U(1)$ group in \ac{QED}, which was
just the electromagnetic charge, $Q$. 
%describe EWK as mark did

The physical gauge bosons observed in experiments on the electroweak
force are obtained by mixing the $W_{\mu}^i$ and $B_{\mu}$ gauge
fields as follows:
\begin{equation}
  \begin{split}
  \PWpm_{\mu}=\frac{1}{\sqrt{2}}\left(\PW^{1}_{\mu}\mp i\PW^{2}_{\mu}\right) \\
  \PZ_{\mu}=\cos\left(\theta_{W}\right)\PW^{3}_{\mu}-\sin\left(\theta_{W}\right)B_{\mu} \\
  A_{\mu}=\sin\left(\theta_{W}\right)\PW^{3}_{\mu}+\cos\left(\theta_{W}\right)B_{\mu},
  \end{split}
\end{equation}
where $A_{\mu}$ is the photon field, $Z_{\mu}$ is the Z boson field
and $W^{\pm}_{\mu}$ are the W boson fields. The Weinberg angle,
$\theta_W$ is given by the coupling strengths of the weak hypercharge
gauge field, $g'$, and the isospin gauge field, $g$:
\begin{equation}
\theta_W = \frac{g}{g^2+g'^2}.
\end{equation}

The $W^{\pm}$ gauge bosons only couple to the left handed component of
the fermion fields. These left-handed components form weak isospin
doublets in both the quark, $Q_L$, and lepton fields $e_L$. The
right-handed components form weak isospin singlets, $Q_R$ and $e_R$
and have $t_3=0$. For the first generation of quarks, the $u$ and $d$
and the first generation of leptons, $e$ and $\nu_e$, the left and
right handed components of the fields are broken down as follows:
\begin{equation}
  \begin{split}
  e_L=\left(\begin{array}{c} \nu_{e~L} \\
  e_L\end{array}\right),~~~
  Q_L=\left(\begin{array}{c} u_L \\
  d_L\end{array}\right),~~~e_R=e_R,~~~Q_R=u_R,d_R,
  \end{split}
\end{equation}
where a subscript $L$ denotes the left-handed component and a
subscript $R$ denotes the right handed component.

Within the quark doublet the charged current interactions of the
$W^{\pm}$ fields act between up and down type quarks. However, the
mass eigenstate of the quarks is not the same as the electroweak
eigenstate. The mixing between these two eigenstates is described by
the \ac{CKM} matrix \cite{Kobayashi:1973fv}. The matrix is diagonally
dominant, meaning the $W^{\pm}$ fields are most likely to produce
interactions of quarks in the same generation, however this allows for
a non-zero intergenerational mixing of the quark fields.

\subsection{Spontaneous symmetry breaking and the Higgs mechanism}

Despite the success of the description of the \SM forces through a
gauge theory, initial iterations did not provide a way for the
fundamental \SM particles to have a gauge invariant mass term in the
lagrangian. This problem was solved through a spontaneous breaking of
the electroweak symmetry that became known as the \emph{Higgs
mechanism}
\cite{Englert:1964et,Higgs:1964ia,Higgs:1964pj,Guralnik:1964eu,Higgs:1966ev,Kibble:1967sv}.
This spontaneous symmetry breaking allowed the vector bosons of the
weak force to obtain mass and provided a way to write gauge invariant
mass terms for the fermions.

A symmetry is spontaneously broken if the ground state of
the vacuum does not share the symmetry of the lagrangian
\cite{Griffiths:111880}. Even though the collection of all states does
share the symmetry, when the theory is in its ground state a
particular vacuum energy must be chosen. This allows terms which are
not gauge invariant to be added to the theory by coupling some of the
fields to a new field with a non-zero vacuum expectation value. This
is achieved within the \SM by introducing a complex scalar $SU(2)$
field with four degrees of freedom, $\phi$, called the Higgs field:
\begin{equation}
\phi=\left(\begin{array}{c}\phi^+ \\ \phi^0 \end{array}\right).
\end{equation}
This is implemented into the theory through an additional term in the
\SM lagrangrian:
\begin{equation}
\mathcal{L}_{Higgs} =
(\mathcal{D}_{\mu}\phi)^{\dag}(\mathcal{D}^{\mu}\phi) - V(\phi),
\end{equation}
where the covariant derivative is chosen to keep the Higgs field
invariant under $SU(2)\times U(1)$ transformations with a weak
hypercharge of $y=\frac{1}{2}$.

% of:
% \begin{equation}
% \mathcal{D}_{\mu}=\partial_{\mu}-\frac{i}{2}g_1W_{\mu}.
% \end{equation}
To spontaneously break the symmetry the potential, $V$, is chosen to
take the form:
\begin{equation}
V(\phi)=-\mu^{2}\phi^{\dag}\phi+\lambda\left(\phi^{\dag}\phi\right)^{2},
\end{equation}
where $\mu^2>0$ and $\lambda>0$. This leads to a potential with a
non-zero expectation value that forms a circle in phase space. This
leads to a continuous set of equivalent minima of which one must be
chosen, resulting in the spontaneous symmetry breaking. By convention a
particular minimum is chosen as:
\begin{equation}
\bra{0}\phi\ket{0}=\left(\begin{array}{c} 0 \\ \sqrt{\frac{\mu^{2}}{2\lambda}} \end{array}\right)=\frac{1}{\sqrt{2}}\left(\begin{array}{c} 0 \\ v \end{array}\right).
\end{equation}
Perturbations about this vacuum expectation value can be parametrised
in the form of four real scalar fields. However, with an appropriate
choice of gauge, three of these degrees of freedom, known as the
\emph{Goldstone bosons} can be set to zero. This leaves one remaining
field, $H$, and perturbations can be written as:
\begin{equation}
  \phi=\left(\begin{array}{c}0 \\ v+H \end{array}\right).
\end{equation}
This can then be inserted into the lagrangian to obtain at leading
order:
\begin{equation}
  \mathcal{L}=\frac{1}{2}\partial_{\mu}H\partial^{\mu}H-\frac{1}{2}\mu^{2}H^{2}+\frac{v^{2}}{8}\left[g_{2}^{2}W_{\mu}^{+}W^{+\mu}+g_{2}^{2}W_{\mu}^{-}W^{-\mu}+\left(g_{1}^{2}+g_{2}^{2}\right)Z_{\mu}Z^{\mu}\right].
\end{equation}
This provides the weak vector bosons $W_{\mu}^{\pm}$ and $Z_{\mu}$
with mass terms $g_2v/2$ and $\frac{v}{2}\sqrt{g_1^2+g_2^2}$
respectively. This also introduces a massive scalar field, $H$, with a
mass $\sqrt{2\mu^2}$, which is the Higgs boson. The aim of providing
the weak vector bosons with mass in a gauge invariant way has
therefore been achieved with the introduction of a Higgs boson. This
Higgs boson has subsequently been discovered by the ATLAS and \CMS
collaborations with a mass of 125~\gev \cite{1207.7214,1207.7235}.

With the existence of a Higgs field, it is also possible to write
gauge invariant mass terms for the fermion fields. These are known as
\emph{Yukawa terms} and take the form:
\begin{equation}
  \mathcal{L}_{Yuk}=y_{f}\left(\bar{f}_{L}\phi f_{R}+\bar{f}_{R}\phi^{\dag}f_{L}\right),
\end{equation}
where $f_L$ is the left-handed component of the fermionic field and
$f_R$ is the right-handed component. The value $y_f$ is the Yukawa
coupling and leads to a fermion mass of $y_fv/\sqrt{2}$. The magnitude
of the mass of the fermion is therefore determined by how strongly it
couples to the Higgs field.

\subsection{Beyond the Standard Model}

% all of SM verified so far, everything found, (requires some
% modifications for neutrino masses)
% but there are bigger problems:

%hierarchy problem

%DM
% show the bullet cluster?

%GUT

\section{Supersymmetry}
\label{sec:susy}

%fundamental spacetime symmetry

% as in Mark's and in primer

%solves above problems

\subsection{The Minimal Supersymmetric Standard Model (MSSM)}

\subsection{Signatures of supersymmetry at the \LHC}

As the \LHC is a hadron collider, the highest cross section SUSY
production processes occur via the strong force
\cite{Martin:1997ns}
\cite{SUSYxsections_NewAspectsof_pp_collisions}. These processes
result in the production of squarks and gluinos, the SUSY particles
with colour charge. In all favoured SUSY models, these relatively
heavy particles decay within the detector to a weakly interacting \LSP,
usually a neutralino \cite{Farrar:1978xj}. In
collisions at the \LHC this will appear as several hard jets with
unbalanced momentum (missing energy). 

\subsection{Simplified models}

%Decays etc.
