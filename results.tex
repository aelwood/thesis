\chapter{Results and interpretations}
\label{chap:results}

The analysis described in the previous chapters has been carried out
on the $12.9~\ifb$ dataset that was introduced in
Sec.~\ref{sec:dataset}, the results are described in this chapter. The
\SM backgrounds have been predicted using the likelihood fit discussed
in Sec.~\ref{sec:likelihood}. The data has then been compared to the expected
background yields and limits are set on the production of an array of
different \SUSY models.

%%%%%%%%%%%%%%%%%%%%
\section{Results}

The total predicted \SM yields for each of the (\HT,\nj,\nb) bins,
integrated over the \MHT dimension, are shown in
Figs.~\ref{fig:mono},~\ref{fig:asym} and \ref{fig:sym} for the
monojet, asymmetric and symmetric jet categories respectively. In the
top panel of each figure the data counts with a representative
statistical uncertainty are shown as black circles with error bars.
The coloured histogram shows the result of the \SM background
prediction with the \TF methods described in
Chapter~\ref{chap:backgroundPred}, the uncertainty of this prediction
is represented with a shaded box (CR-only fit uncertainty). The
predictions are split into the \znunu, \QCD multijet and other
remaining \SM backgrounds. On the bottom panel of each plot the
significance of the deviation of the data from the predicted \SM
background is plotted. The red circles show the deviation of the
control region only background fit (Eq.~\ref{eq:controlLikelihood}),
while the blue circles show the deviation of the fit that includes the
signal region (Eq.~\ref{eq:total_likelihood}). The deviation is
represented as a \emph{pull} that is defined as the number of observed
events, minus the number of predicted events, divided by the total
uncertainty.

Along with the background prediction in each of the (\HT,\nj,\nb)
categories, simulation is used to predict the \MHT shapes. The result
of this prediction for a series of representative bins is shown in
Fig.~\ref{fig:mht-templates}. The data yields and statistical errors
are displayed by the black markers. The \MHT shape taken from
simulation normalised based on the results from the control region
only fit is displayed as the green histogram.

Overall, no significant deviation from the \SM backgrounds is observed
in the data. The results are well described by a \SM-only hypothesis.
There are a few large pulls $\sim 3\sigma$ that are observed in the
control-region only fit. However, after a fit including the signal
region is carried out the pulls are significantly reduced. This
suggests that such effects are properly covered by systematic
uncertainties.

\begin{figure}[!h]
  \begin{center}
    \includegraphics[width=0.9\textwidth]{figs/analysis/results/summaryPlot_Monojet_prefit_overlay_fit_b}
    \caption{The total event yields in data (solid black circles)
      and the \SM expectations with their associated uncertainties (black
      histogram with shaded band) as a function of
      \nb and \HT for the monojet topology ($\njet = 1$) in the
      signal region. Under this is the significance of deviations
      (pulls) observed in data with respect to the \SM expectations
      from the fit with only the control regions (red circles) and a
      full fit including the signal region (blue circles).}
    \label{fig:mono}
  \end{center}
\end{figure}

\begin{figure*}[!h]
  \begin{center}
    \includegraphics[angle=90,width=0.7\textwidth]{figs/analysis/results/summaryPlot_Asymmetric_prefit_overlay_fit_b}
    \caption{The total event yields in data (solid black circles)
      and the \SM expectations with their associated uncertainties (black
      histogram with shaded band) integrated over \MHT as a function of
      \nj,\nb and \HT for the asymmetric topology in the
      signal region. Under this is the significance of deviations
      (pulls) observed in data with respect to the \SM expectations
      from the fit with only the control regions (red circles) and a
      full fit including the signal region (blue circles).}
    % \caption{(Top panel) Event yields observed in data (solid circles)
    %   and SM expectations with their associated uncertainties (black
    %   histogram with shaded band) from a CR-only fit, integrated over
    %   \MHT, as a function of \njet, \nb, and \scalht for the
    %   asymmetric topology in the signal region. (Bottom panel). The
    %   significance of deviations (pulls) observed in data with respect
    %   to the SM expectations from the CR-only (red circles) and full
    %   fit (blue circles). The pulls are indicative only and cannot be
    %   considered independently.}
    \label{fig:asym}
  \end{center}
\end{figure*}

\begin{figure*}[!h]
  \begin{center}
    \includegraphics[angle=90,width=0.7\textwidth]{figs/analysis/results/summaryPlot_Symmetric_prefit_overlay_fit_b}
    % \caption{(Top panel) Event yields observed in data (solid circles)
    %   and SM expectations with their associated uncertainties (black
    %   histogram with shaded band) from a CR-only fit, integrated over
    %   \MHT, as a function of \njet, \nb, and \scalht for the
    %   symmetric topology in the signal region. (Bottom panel). The
    %   significance of deviations (pulls) observed in data with respect
    %   to the SM expectations from the CR-only (red circles) and full
    %   fit (blue circles). The pulls are indicative only and cannot be
    %   considered independently.}
    \caption{The total event yields in data (solid black circles)
      and the \SM expectations with their associated uncertainties (black
      histogram with shaded band) integrated over \MHT as a function of
      \nj,\nb and \HT for the symmetric topology in the
      signal region. Under this is the significance of deviations
      (pulls) observed in data with respect to the \SM expectations
      from the fit with only the control regions (red circles) and a
      full fit including the signal region (blue circles).}
    \label{fig:sym}
  \end{center}
\end{figure*}

\begin{figure*}[tbhp]
  \begin{center}
  \subfloat[Symmetric topology, medium \HT, low \nj]{
    \includegraphics[width=0.49\textwidth]{figs/analysis/results/mhtShape_eq0b_eq2j_600_800_fit_b.pdf}
    }~~
  \subfloat[Asymmetric topology, low \HT]{
    \includegraphics[width=0.49\textwidth]{figs/analysis/results/mhtShape_eq1b_eq3a_300_350_fit_b.pdf}
    }\\
  \subfloat[Symmetric topology, high \nj and \HT]{
    \includegraphics[width=0.49\textwidth]{figs/analysis/results/mhtShape_eq0b_ge5j_800_Inf_fit_b.pdf}
    }~~
  \subfloat[Symmetric topology, high \nj, \nb and \HT]{
    \includegraphics[width=0.49\textwidth]{figs/analysis/results/mhtShape_ge3b_ge5j_800_Inf_fit_b.pdf}
    }
  \end{center}
  \caption{The total event yields in data (solid black circles) and
  the \SM expectations with their associated uncertainties (green
  histogram with shaded band) as a function of \MHT for events in the
  signal region for four representative signal region categories. The
  final bin of each histogram is an overflow bin. Under this is the
  significance of deviations (pulls) observed in data with respect to
  the \SM expectations. \label{fig:mht-templates} 
  }
\end{figure*}

% some representative systematics or not...

%%%%%%%%%%%%%%%%%%%%
\section{Interpretation of the results}
\label{sec:signalModel}

% introduce the signal models used

% feynman diagrams of the models

\subsection{Uncertainties on signal models}

% list the uncertainties as in the AN

\subsection{Exclusion limits}

% describe and show the limits

