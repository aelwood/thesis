%% Title
\titlepage[Imperial College London\\Department of Physics]{%
  A thesis submitted to Imperial College London\\ for the degree of Doctor of Philosophy}

%% Abstract
\begin{abstract}%[\smaller \thetitle\\ \vspace*{1cm} \smaller {\theauthor}]
  %\thispagestyle{empty}
  An inclusive search for supersymmetry with jets and missing
  transverse energy is presented. Data from $\sqrt{s}=13~\tev$
  $pp$-collisions with a total integrated luminosity of $12.9~\ifb$
  delivered by the LHC and collected by the CMS detector are analysed.
  The dominant quantum chromodynamic multijet background is strongly suppressed with several
  kinematic variables, which are also used to discriminate between
  standard model and supersymmetric processes. The observed events are
  found to be compatible with the expected contributions from standard
  model processes. This result is interpreted in the context of
  simplified supersymmetric models of gluino and third-generation
  squark production. The mass of the gluino, bottom squark and top squark
  are excluded to 1775, 1025 and 875~\gev respectively.

  In preparation for the collection of $\sqrt{s}=13~\tev$ data by CMS,
  the jet algorithm for the Level-1 trigger is upgraded. The new
  algorithm allows for dynamic pileup subtraction and takes advantage
  of hardware upgrades to the trigger. The performance of variations
  of pileup subtraction for upgrade algorithm are evaluated and the
  most promising algorithm, \emph{chunky-donut subtraction}, is
  chosen. The algorithm is found to give a significant performance
  improvement and has been used to collect data from 2016 onwards.
\end{abstract}


%% Declaration
\begin{declaration}
  This dissertation is the result of my own work, except where
  explicit reference is made to the work of others. To produce the
  results in this thesis I worked in collaboration with other members of
  the $\alphat$ analysis group and the trigger studies group, who were
  predominantly based at Imperial College London. Figures from CMS
  preliminary and unpublished results are labelled ``CMS
  Preliminary''. Figures from CMS publications are labelled ``CMS'' or
  ``CMS Simulation'', when the data is only taken from simulation. This
  thesis has not been submitted for another qualification to this or
  any other university. 
  \vspace*{1cm}
  \begin{flushright}
    Adam Elwood
  \end{flushright}
  \vspace*{1cm}
  {\it The copyright of this thesis rests with the author and is made
  available under a Creative Commons Attribution Non-Commercial No
  Derivatives licence. Researchers are free to copy, distribute or
  transmit the thesis on the condition that they attribute it, that
  they do
  not use it for commercial purposes and that they do not alter,
  transform or build upon it. For any reuse or redistribution,
  researchers must make clear to others the licence terms of this
  work}
\end{declaration}


%% Acknowledgements
\begin{acknowledgements}
It has been a great privilege to work on the frontier of fundamental
physics research with many talented researchers. I would first like to
thank Alex Tapper for his excellent supervision and guidance
throughout my studies. The help and expertise of the
other researchers in the group has also been invaluable. I would like to
thank everyone within the $\alphat$ analysis group, particularly Rob
Bainbridge, Stefano Casasso and Bjoern Penning. Also, for their enthusiastic
guidance during the development of a new trigger algorithm, I'd like to
thank Jad Marrouche and Andrew Rose. Additionally, I'd like to thank
all my fellow students and friends for providing support and much
needed distraction along the way, including (but not limited to): Mark
Baber, Matthew Citron, Louie Corpe, Patrick Dunne, Christian Laner,
Lucien Lo, Federico Redi and Adinda de Wit.

I would also like to extend my thanks to everyone in the Imperial
College HEP group, who provided a supportive and interesting community
in which to carry out research. Also, thanks to the STFC,
who provided the funding that allowed me to conduct my research and
have a great experience for two years living and working at CERN.

Finally, I would like to thank my family, particularly my parents, for
providing the immense support required for me to make it this far.
Without the keen interest they displayed in my subject through all
stages of my study I wouldn't be in the position I am now. Thanks
also to Giuditta for tolerating and supporting me as I jumped the
final hurdles.
\end{acknowledgements}


%% Preface
% \begin{preface}
%   This thesis describes my research on various aspects of the \LHCb
%   particle physics program, centred around the \LHCb detector and \LHC
%   accelerator at \CERN in Geneva.
%
%   \noindent
%   For this example, I'll just mention \ChapterRef{chap:SomeStuff}
%   and \ChapterRef{chap:MoreStuff}.
% \end{preface}
%
%% ToC


%% Strictly optional!
\frontquote{Traditional scientific method has always been, at the
very best, 20-20 hindsight. It's good for seeing where you've been.
It's good for testing the truth of what you think you know, but it
can't tell you where you ought to go.}%
{Robert M. Pirsig,   \emph{Zen and the art of motorcycle maintenance}}


%\frontquote{%
%Come, let us hasten to a higher plane,\\
%Where dyads tread the fairy fields of Venn,\\
%Their indices bedecked from one to n,\\
%\vspace{0.2cm}
%Commingled in an endless Markov chain!\\
%Come, every frustum longs to be a cone,\\
%And every vector dreams of matrices.\\
%Hark to the gentle gradient of the breeze:\\
%\vspace{0.2cm}
%It whispers of a more ergodic zone.\\
%In Riemann, Hilbert, or in Banach space\\
%Let superscripts and subscripts go their ways.\\
%Our asymptotes no longer out of phase,\\
%\vspace{0.2cm}
%We shall encounter, counting, face to face.\\
%I'll grant thee random access to my heart,\\
%Thou'lt tell me all the constants of thy love;\\
%And so we two shall all love's lemmas prove,\\
%\vspace{0.2cm}
%And in our bound partition never part.\\
%For what did Cauchy know, or Christoffel,\\
%Or Fourier, or any Boole or Euler,\\
%Wielding their compasses, their pens and rulers,\\
%\vspace{0.2cm}
%Of thy supernal sinusoidal spell?\\
%Cancel me not -- for what then shall remain?\\
%Abscissas, some mantissas, modules, modes,\\
%A root or two, a torus and a node:\\
%\vspace{0.2cm}
%The inverse of my verse, a null domain.\\
%Ellipse of bliss, converge, O lips divine!\\
%The product of our scalars is defined!\\
%Cyberiad draws nigh, and the skew mind\\
%\vspace{0.2cm}
%Cuts capers like a happy haversine.\\
%I see the eigenvalue in thine eye,\\
%I hear the tender tensor in thy sigh.\\
%Bernoulli would have been content to die,\\
%Had he but known such $a^2 cos(2\phi)$.
%\vspace{0.2cm}}
%{Trurl in response to
%Klapaucius in Lem's Cyberiad}
%% I don't want a page number on the following blank page either.
\thispagestyle{empty}

\tableofcontents
