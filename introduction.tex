\chapter{Introduction}
\label{chap:introduction}

%% Restart the numbering to make sure that this is definitely page #1!
\pagenumbering{arabic}

%% Note that the citations in this chapter use the journal and
%% arXiv keys: I used the SLAC-SPIRES online BibTeX retriever
%% to build my bibliography. There are also quite a few non-standard
%% macros, which come from my personal collection. You can have them
%% if you want, or I might get round to properly releasing them at
%% some point myself.

\chapterquote{They don't think it be like it is, but it do}%
{Oscar Gamble}

% \section{}
% \label{}

Modern physics has now reached a point in which our fundamental
understanding can be broken down into two separate arenas. At large
length scales, where gravity is dominant, the theory of \GR is incredibly successful in reproducing experimental
observations.  However, to provide a description of the subatomic
constituents of matter and the three other fundamental forces we rely
on a quantum field theory, the \SM of particle
physics. However, at very high energy densities where both theories
are relevant, our understanding breaks down. The main aim of
fundamental physics research is therefore the reconcilation of quantum
field theory with general relativity.
  
The \SM is one of the most successful scientific theories to date
\cite{Salam:1964ry,Glashow:1961tr,PhysRevLett.19.1264}. It makes
predictions about the physical world that have consistently stood up
to experimental scrutiny, culminating in the discovery of a $125$\gev
particle consistent with a Higgs boson at the \LHC in 2012~\cite{1207.7214,1207.7235}.  Despite
its successes, the \SM does not provide a description for two
significant experimental anomalies. From astronomical observations,
it can be inferred that \SM particles cannot solely account for the
total gravitational behaviour of various objects in the universe. This
anomaly can be explained by introducing a new form of weakly
interacting particle, known as ``\DM''. On top of this, it is
observed that the rate of the expansion of the universe is increasing,
implying the existence of ``dark energy''. A rough calculation of the
scale of dark energy predicted by the \SM yields a result that is many
of order of magnitude different from the observed value. This implies
a signicant lack of understanding of the origin of this phenomena.
Additionally, issues of fine tuning are present within the \SM itself.
Along with the irreconcilability of the \SM with \GR, these problems
point towards the existence of a more fundamental theory of nature
beyond the \SM.

One popular extension of the \SM is the introduction of a new broken
spacetime symmetry between fermions and bosons, known as supersymmetry
(\SUSY). Initially motivated from a mathematical standpoint, \SUSY
models can provide a candidate for dark matter, solve the Higgs
hierarchy problem and also unify the strong, weak and electromagnetic
forces at the ``GUT'' scale, which is not possible in the \SM. To
convincingly solve these problems, \SUSY is expected to exhibit itself
close to the electroweak scale of the \SM. If this is the case, there
is a significant chance that supersymmetric particles will be produced
at the \LHC. This will be further explained in
Chapter~\ref{chap:theory}.

In Run~1 of the \LHC $20\ifb$ of collision data at an $8\tev$ centre of
mass energy was collected by \CMS. The analysis of this data, however,
has not resulted in any observation of \SUSY production. With Run~2 of
the LHC, that began in 2015, the collision energy has been almost
doubled to 13\tev. The data taken in this new run therefore hold the
best chance yet for the discovery of electroweak scale \SUSY.

In this thesis will be presented the details and results of a search
for \SUSY in Run~2 data collected by the \CMS detector at the \LHC.  The
analysis presented looks for \SUSY signatures in a final state with
hadronic jets and missing transverse momentum. The results of this
search are interpreted in a series of simplified \SUSY and \DM models.

%To do at the end:
Outline 

Declaration

Future

%
% \begin{equation} \I \pdByd{}{t} \colvector{a \\ b} = \underbrace{%
% \twomatrix{ M_{11}-\frac{\I}{2}\Gamma_{11} &
% M_{12}-\frac{\I}{2}\Gamma_{12} } {
% M_{12}^\ast-\frac{\I}{2}\Gamma_{12}^\ast &
% M_{22}-\frac{\I}{2}\Gamma_{22} } }_{\boldmatrix{H}} \colvector{a \\
% b} .  \end{equation}
