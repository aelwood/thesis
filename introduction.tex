\chapter{Introduction}
\label{chap:introduction}

%% Restart the numbering to make sure that this is definitely page #1!
\pagenumbering{arabic}

%% Note that the citations in this chapter use the journal and
%% arXiv keys: I used the SLAC-SPIRES online BibTeX retriever
%% to build my bibliography. There are also quite a few non-standard
%% macros, which come from my personal collection. You can have them
%% if you want, or I might get round to properly releasing them at
%% some point myself.

% \chapterquote{Listen, Morty, I hate to break it to you but what people
% call ``love'' is just a chemical reaction that compels animals to breed.
% It hits hard, Morty, then it slowly fades, leaving you stranded in a
% failing marriage. I did it. Your parents are gonna do it. Break the
% cycle, Morty. Rise above. Focus on science.}
% {Rick Sanchez, Rick and Morty}

% \section{}
% \label{}

Modern physics has now reached a point in which our fundamental
understanding can be broken down into two separate arenas. At large
length scales, where gravity is dominant, the theory of \GR
\cite{1914ZMP63215E} is incredibly successful in reproducing
experimental observations \cite{Will:2014kxa}.  However, to provide a
description of the subatomic constituents of matter and the three
other fundamental forces, we rely on a quantum field theory, the \SM
of particle physics
\cite{Salam:1964ry,Glashow:1961tr,PhysRevLett.19.1264}. At very high
energy densities, where both theories are relevant, our understanding
breaks down \cite{Weinberg:1980gg}. The main aim of fundamental
physics research is therefore the reconcilation of quantum field
theory with general relativity. 

One way to approach this is to explore higher energy scales, above the
mass scales of the \SM, but before \GR becomes relevant. To this end, a
giant proton synchrotron, the \LHC, was built at CERN near Geneva,
Switzerland \cite{Evans:2008zzb}. The \LHC is designed to collide
protons at record breaking energies. Around these collision
points are built various detectors that explore the results of these
high energy collisions. One of the two general purpose detectors
designed for searching for a wide range of phenomena is the \CMS
detector \cite{Chatrchyan:2008aa}. It is the exploration of the new
energies beyond the \SM with \CMS that is the subject of this thesis.

The \SM is one of the most successful scientific theories to date. It
makes predictions about the physical world that have consistently
stood up to experimental scrutiny, culminating in the discovery of a
$125$\gev particle consistent with a Higgs boson at the \LHC in
2012~\cite{1207.7214,1207.7235}.  Despite its successes, the \SM does
not provide a description for some significant experimental anomalies.
From astronomical observations, it can be inferred that \SM particles
cannot solely account for the total gravitational behaviour of various
objects in the Universe
\cite{Kapteyn:1922zz,Oort:436532,Markevitch:2003at,2012Natur.487..202D,Ade:2015xua,0067-0049-180-2-225}.
This anomaly can be explained by introducing a new form of weakly
interacting particle, known as \emph{\DM}. On top of this it is
observed that the rate of the expansion of the universe is increasing,
implying the existence of \emph{dark energy}
\cite{Weinberg:1988cp,Riess:1998cb}.  A rough calculation of the scale
of dark energy predicted by the \SM yields a result that is many of
order of magnitude different from the observed value.  This implies a
signicant lack of understanding of the origin of this phenomena.
Additionally, fine tuning problems are present within the \SM itself.
Along with the irreconcilability of the \SM with \GR, these issues
point towards the existence of a more fundamental theory 
that goes beyond the \SM. 

One popular way to extend the \SM is to introduce a new broken
spacetime symmetry between fermions and bosons, known as \SUSY
\cite{Martin:1997ns}.  Initially motivated with purely mathematical
arguments, \SUSY models can provide a candidate for \DM, solve the
Higgs hierarchy problem and also unify the strong, weak and
electromagnetic forces at the \ac{GUT} scale, which is not possible
in the \SM. To convincingly solve these problems, \SUSY is expected to
exhibit itself close to the electroweak scale of the \SM. If this is
the case, there is a significant chance that supersymmetric particles
will be produced at the \LHC. %This will be further explained in
%Chapter~\ref{chap:theory}. 

%In Run~1 the \LHC delivered $19.7~\ifb$ of collision data at an
In the first run of the \LHC, Run~1, it delivered a large dataset at
$7$ and $8~\tev$ centre of mass energies. The analysis of these data,
however, has not resulted in any observation of \SUSY. With Run~2 of
the \LHC, that began in 2015, the centre of mass collision energy has
been increased to 13\tev. The data taken in this new run therefore
hold the best chance yet for the discovery of electroweak scale \SUSY.

This thesis presents the description and results of a search for \SUSY
in data collected by the \CMS detector at the \LHC during
proton-proton collisions with a centre of mass energy $\sqrt{s} =
13~\tev$.  The analysis presented looks for \SUSY signatures in a
final state with hadronic jets and missing transverse momentum. 
% The
% results of this search are then interpreted in a series of simplified
% \SUSY and \DM models.

