\chapter{Introduction}
\label{chap:introduction}

%% Restart the numbering to make sure that this is definitely page #1!
\pagenumbering{arabic}

%% Note that the citations in this chapter use the journal and
%% arXiv keys: I used the SLAC-SPIRES online BibTeX retriever
%% to build my bibliography. There are also quite a few non-standard
%% macros, which come from my personal collection. You can have them
%% if you want, or I might get round to properly releasing them at
%% some point myself.

\chapterquote{They don't think it be like it is, but it do}%
{Oscar Gamble}

% \section{}
% \label{}

Modern physics has now reached a point in which our fundamental
understanding can be broken down into two separate arenas. At large
scales, where gravity is dominant, the theory of General Relativity
(GR) is incredibly successful in reproducing experimental
observations.  However, to provide a description of the subatomic
constituents of matter and the three other fundamental forces we rely
on the Standard Model (SM) of particle physics. The main aim of
fundamental physics research is now the reconcilation of these two
very successful theories.

The SM is the most accurate scientific theory to date
\cite{Salam1964}\cite{Glashow1961}\cite{Weinberg1967}. It makes
predictions about the physical world that have consistently stood up
to experimental scrutiny, culminating in the discovery of a
$125$~GeV particle consistent with a Higgs boson at the Large Hadron
Collider (\LHC) in 2012~\cite{ATLASHiggs2012}\cite{CMS2012HiggsPaper}.
Despite its successes, the SM does not provide a description for two
significant experimental anomalies. From astrophysical observations of the
galaxies it can be inferred that SM
particles cannot solely account for the total gravitational behaviour
of various objects in the Universe. This anomaly implies the existence
of ``dark matter''. On top of this, it is observed that the expansion
of the universe is increasing, implying the existence of 
``dark energy''. Additionally, issues of fine tuning are present
within the theory itself. Along with the irreconcilability of the SM
with GR, these problems point towards the existence of a more
fundamental theory of nature beyond the SM.

One popular extension of the SM is the introduction of a new broken
spacetime symmetry between fermions and bosons, known as supersymmetry
(SUSY). Initially motivated from a mathematical standpoint, SUSY
models can provide a candidate for dark matter, solve the Higgs
hierarchy problem and also unify the
strong, weak and electromagnetic forces at the ``GUT''
scale, which is not possible in the SM. To convincingly solve
these problems, SUSY is expected to exhibit itself close to the electroweak scale
of the SM. If this is the case, there is a significant chance that
supersymmetric particles will be produced at the \LHC.

Run~1 of the LHC has not resulted in any observation of SUSY
production as of yet. With Run~2 of the LHC, that began in 2015, the
collision energy has been almost doubled. The data taken in this new
run therefore hold the best chance yet for the discovery of electroweak scale
SUSY. If it exists at a scale that solves the hierarchy problem with
minimal fine tuning, it should be seen at the LHC.

%To do at the end:
Outline 

Declaration

Future

%
% \begin{equation} \I \pdByd{}{t} \colvector{a \\ b} = \underbrace{%
% \twomatrix{ M_{11}-\frac{\I}{2}\Gamma_{11} &
% M_{12}-\frac{\I}{2}\Gamma_{12} } {
% M_{12}^\ast-\frac{\I}{2}\Gamma_{12}^\ast &
% M_{22}-\frac{\I}{2}\Gamma_{22} } }_{\boldmatrix{H}} \colvector{a \\
% b} .  \end{equation}
