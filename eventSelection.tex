\chapter{Event selection and analysis strategy}
\label{chap:selection}

% introduce analysis and what the next chapters are about

%%%%%%%%%%%%%%%%%%%%
\section{Challenges for a hadronic \BSM search}
\label{sec:challenge}

\subsection{The QCD multijet background}

\subsection{Backgrounds from standard model processes with genuine
\MET}

%%%%%%%%%%%%%%%%%%%%
\section{QCD multijet suppression with topological variables}
\label{sec:challenge}

\subsection{The \alphat variable}
%introduce the variables and why they're useful

%not finished, from 18 month report:
Due to the very high cross sections of QCD processes at the LHC, there
is a large multijet background. In the case that one of the jets is
mismeasured, this results in fake missing energy, the signature left
by SUSY production. To negate this effect, the dimensionless variable
$\alpha_T$ is introduced
\cite{AlphaTproposalCMS:2008vya,AlphaTproposalPhysRevLett.101.221803}.
For a dijet system it is defined as: 
\begin{equation}
\alpha_T=\frac{E_T^{j_2}}{M_T}, \end{equation} 
where $E_T^{j_2}$ is
the transverse energy of the lower energy jet and
$M_T=\sqrt{H_T^2-\cancel{H}_T^2}$ is the invariant mass of the dijet
system. It is constructed from the jet characterising variables \HT
and \MHT
with $n_{jet}$ jets, $j_i$, with transverse momentum
$\vec{p}_T^{j_i}$ and transverse energy $E_T^{j_i}$. For events with
more than two jets, a pseudo dijet system is formed by combining jets.
The system chosen is one that minimises $|\Delta H_T|$, this is the
difference between the $E_T$ of each pseudo jet, where $E_T$ is the
scalar sum of the transverse energies of all the jets in each pseudo
jet. This leads to a generalised form of $\alpha_T$
\cite{AlphaT8TeVChatrchyan:2013lya}: 
\begin{equation}
\alpha_T=\frac{1}{2}\times\frac{H_T-\Delta
H_T}{\sqrt{H_T^2-\cancel{H}_T^2}} \end{equation} 
Balanced and
mismeasured multijet events have values of $\alpha_T\leq0.5$, while
events with genuine missing energy have values of $\alpha_T>0.5$. By
choosing an appropriate cut above $0.5$ it is possible to reduce the
multijet background to a negligible amount, this is evident in
Fig.~\ref{fig:alphaT}. 

% put more upt to date one here
% \begin{figure}
% 	\begin{center}
% 		\includegraphics[width=0.8\linewidth]{}%alphaT1_bkgd}
% 	\end{center}
%   \caption{The $\alpha_T$ values for events with $H_T>375$ GeV and 2
%   to 3 jets that pass all other cuts imposed in the $\alpha_T$
%   analysis. The green dotted line shows the expected multijet QCD
%   background that can be removed with an appropriate cut on $\alpha_T$
%   \cite{AlphaT8TeVChatrchyan:2013lya}}
% 	\label{fig:alphaT}
% \end{figure}

\subsection{The \bdphi variable}
%put the stuff about bDPhi over dPhi as in the 

\subsection{\MHT/\MET}

%%%%%%%%%%%%%%%%%%%%
\section{Physics objects}

%%%%%%%%%%%%%%%%%%%%
\section{Trigger strategy}

%%%%%%%%%%%%%%%%%%%%
\section{Event selection and categorisation}

\subsection{Pre-selection}

%work in progress from 18 month:
The only genuine source of missing transverse energy ($\cancel{E}_T$)
in the SM is electroweak neutrino production. In these cases, with the
exception of a $Z$ decaying via a pair of neutrinos, an associated
lepton is simultaneously produced. This background is minimised by
vetoing any events with isolated \footnote{A particle is isolated if
the energy of other particles within a cone of
$R\equiv\sqrt{(\Delta\phi)^2+(\Delta\eta)^2}=0.3$, where $\phi$ is the
azimuthal angle and $\eta$ the pseudorapidity, do not add up to a
significant proportion of the particle's momentum, typically
$\sim10$\%} leptons of $p_T>10$~GeV. To ensure a fully hadronic final
state there is also a veto on photons of $p_T>25$~GeV. Further, to
reduce the ``lost lepton'' backgrounds from W~+~jets and $t\bar{t}$,
events containing single isolated tracks with $p_T >10$~GeV and
$|\eta| < 2.5$ are vetoed.

Events are also required to contain at least one $p_T>100$~GeV and one
$p_T>40$~GeV jet, where the jets are well reconstructed in the central
region, $|\eta|<3$. If any jets fall outside the $\eta$ range, the
event is vetoed. Significant hadronic activity is selected by
requiring $H_T>200$~GeV. Events are categorised based on the number of
jets, the number of jets with a reconstructed b-quark and the value of
$H_T$. 

\subsection{The signal region}

\subsection{The control regions}

\subsection{Event categorisation}

%%%%%%%%%%%%%%%%%%%%
\section{Corrections to simulation}

\subsection{Trigger efficiencies}

\subsection{Scale factors}

%%%%%%%%%%%%%%%%%%%%
%\section{Further optimisation of event selection}
% something about minChi here?
