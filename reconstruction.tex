\chapter{Event reconstruction and simulation}
\label{chap:reconstruction}

After the result of proton collisions have been observed in the
various subdetectors that comprise \CMS, each of these events must be
reconstructed in a way that allows for analysis of the physical
processes occurring in the collision. Information from the
subdetectors is used to infer the presence of different particles
produced in the collision. To help understand the different physical
processes that may occur during \LHC proton collisions \MC simulations
of events are utilised. The reconstruction algorithms and simulations
that are relevant to a search for supersymmetry are described in this
Chapter.

\section{Tracks and vertices}
\label{sec:tracks_reco}

As charged particles pass through the \CMS tracker they leave energy
deposits, known as ``hits'', in each layer. These
hits are reconstructed as tracks with the \ac{CTF}
algorithm \cite{Chatrchyan:2014fea}. This algorithm tries to associate
hits that belong to a single charged particle. This allows the
determination of the track left by the charged particle its curvature
through the magnetic field can be measured. The steps of the algorithm
are as follows:

\begin{itemize}
\item{Initially two or three hits in the inner layers of the tracker
are used to produce seeds for initial track candidates. Quality
selections are applied on the selected hits to reduce the number of fake tracks.} 
\item{Each seed is extrapolated along the expected trajectory using a
Kalman filter \cite{Fruhwirth:1987fm} with a helical tracking
hypothesis.  This allows the seeds to be associated with hits in an
outer tracker layer.} 
\item{The extrapolation is carried out recursively into the subsequent
tracker layers until the outer-most layer, or another
stopping condition, is reached.} 
\item{Of the tracking candidates that are found, additional quality
criteria are required to reject fake tracks.}
\item{This series of steps is repeated up to six times with the hits
associated to identified tracks removed after each iteration. }
\end{itemize}

Track reconstruction efficiencies for a variety of charged particles
are shown in Fig.~\ref{fig:tracks_reco} as a function of \pt and
$\eta$. In the central region of the detector all particles with a \pt
from 10 to 100~\gev are reconstructed with a 90-100\% precision. In the
forward detector region this efficiency remains above $\sim80\%$.
%fake rate? leave out for now...

\begin{figure}
\begin{center}
\includegraphics[width=0.8\linewidth]{figs/reconstruction/trackerPerformance} \end{center}
\caption{Efficiencies of track reconstruction for different charged
particles as a function of \pt and $\eta$. Muons are shown at the top,
pions in the middle and electrons at the bottom. The barrel,
transition and endcap regions are defined by the $\eta$ intervals of
0-0.9, 0.9-1.4 and 1.4-2.5 respectively.  For all the tracks
``high-purity'' quality requirements are made
\cite{Chatrchyan:2014fea}}
\label{fig:tracks_reco} \end{figure}

After charged particle tracks are reconstructed they can be used to
reconstruct the positions, known as interaction vertices, of the different
proton-proton in the event. Tracks are required to originate from a
region that is compatible with the \LHC ``beamspot'', the area in
which the proton beams cross and collisions occur. The $z$ coordinates
of tracks at the closest point of approach to the beamspot are taken
as an input for the deterministic annealing clustering algorithm
\cite{726788:DA}. This finds the most probable vertex positions and
assigns each track to a vertex. The final $x$,$y$,$z$ position of
these vertices is then found using the adaptive vertex fitter 
\cite{Waltenberger:2008zz}. This assigns the most probably position of
the vertex for the given set of input tracks. Quality criteria are
then applied to reject fake vertices. They are chosen in such a way as
to remain efficient for real vertices that typically have a large
number of tracks compatible with them. Finally, the ``primary vertex''
is determined as the vertex with tracks that have the greatest scalar
sum of \pt. Other vertices are then initially attributed to \PU.
However, vertices that are displaced from the initial proton collision
are common signatures of unstable particles that decay within the
detector, such as $b$-hadrons. These can be found in subsequent levels
of reconstruction.

The primary vertex efficiency as a function of the number of tracks in
a cluster can be seen in Fig.~\ref{fig:vertex_reco}. The efficiency increases to
close to $>99\%$ as long as there are a reasonable number of tracks in
an event. The vertex resolution reaches $\sim10~\mu$m in $x,y$ and
$\sim10~\mu$m in $z$ with $>40$ reconstructed tracks.

\begin{figure}
\begin{center}
\includegraphics[width=0.5\linewidth]{figs/reconstruction/vertexPerformance} \end{center}
\caption{ The vertex reconstruction efficiency as a function of the
number of tracks originating from the vertex. Measured in data and
simulation for $\sqrt{s}=7~\tev$ proton collisions.
\cite{Chatrchyan:2014fea}}
\label{fig:vertex_reco} \end{figure}

\section{Particle flow}
\label{sec:pflow_reco}

Each subdetector of \CMS contributes complimentary information
about the different types of particles that pass through the detector
as a whole. This complementary is exploited in identifying the
different types of particle with the \PF algorithm
\cite{CMS-PAS-PFT-09-001,CMS-PAS-PFT-10-001,CMS-PAS-PFT-10-002}. As
\CMS has accurate momentum resolution in the tracker and a
high granularity \ECAL this algorithm allows both to augment the measurement of
objects in the \HCAL. This then allows calibrations specific to
charged and neutral hadrons to be applied. 

The \PF algorithm searches for a set of individual particles that are
known as ``\PF candidates''. They are then classified as charged or
neutral hadrons, photons, muons or electrons. The set of \PF
candidates can then be utilised to calculate other event level
variables, such as for jet reconstruction described in
Sec.~\ref{sec:jets_reco}.

how it works

\section{Electrons and photons}
\label{sec:electrons_reco}

both the same, lose energy before calorimeter

cluster reconstruction

\section{Muons}
\label{sec:muons_reco}

don't lose much energy


\section{Jets}
\label{sec:jets_reco}

\section{Missing transverse energy (MET)}
\label{sec:met_reco}

\section{Monte Carlo (MC) simulation}
\label{sec:mc_reco}

