\chapter{Event reconstruction and simulation}
\label{chap:reconstruction}

After the result of proton collisions have been observed in the
various subdetectors that comprise \CMS, each of these events must be
reconstructed in a way that allows for analysis of the physical
processes occurring in the collision. Information from the
subdetectors is used to infer the presence of different particles
produced in the collision. To help understand the different physical
processes that may occur during \LHC proton collisions \MC simulations
of events are utilised. The reconstruction algorithms and simulations
that are relevant to a search for supersymmetry are described in this
Chapter.

\section{Tracks and vertices}
\label{sec:tracks_reco}

As charged particles pass through the \CMS tracker they leave energy
deposits, known as hits, in each layer. These
hits are reconstructed as tracks with the \ac{CTF}
algorithm \cite{Chatrchyan:2014fea}. This algorithm associates all
hits belonging to a single charged particle and allows the
determination of its curvature through the magnetic field.

\begin{itemize}
\item{Initially two or three hits in the inner layers of the tracker
are used to produce seeds for initial track candidates. Quality
selections are applied to reduce the number of fake tracks.} 
\item{The seed is extrapolated along the expected trajectory using a
Kalman filter \cite{Fruhwirth:1987fm} with a helical tracking
hypothesis.  This allows the seeds to be associated with a hit in an
outside tracker layer.} 
\item{This is carried out recursively into the subsequent tracker
layers until the outer-most layer is reached or another stopping
condition.} 
\item{With all the remaining tracking candidates, additional quality
criteria are required to reject fake tracks.}
\item{This series of steps is repeated up to six times with the hits
associated to identified tracks removed after each iteration. }
\end{itemize}


efficiency and resolution (show plots?)

vertex reconstruction

\section{Electrons and photons}
\label{sec:electrons_reco}

\section{Muons}
\label{sec:muons_reco}

\section{Particle flow}
\label{sec:pflow_reco}

\section{Jets}
\label{sec:jets_reco}

\section{Missing transverse energy (MET)}
\label{sec:met_reco}

\section{Monte Carlo (MC) simulation}
\label{sec:mc_reco}

